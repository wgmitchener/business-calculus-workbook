\Section{Roots and fractional exponents}
\label{sec:roots-frac-exp}

\Subheading{Meaning of the root symbol}

The principal $n$-th root of a real number $a$, written
$\sqrt[n]{a}$,
is the real number $z$ that solves
$z^n = a$ and has the same sign as $a$.
The term \emph{radical} is a synonym for the root symbol.
The number $a$ is called the \emph{radicand}.
The number $n$ is called the \emph{index} or \emph{degree} of the root.
We generally require the index to be \emph{positive}.

A plain root symbol $\sqrt{a}$ means the square root $\sqrt[2]{a}$.

The 3rd root $\sqrt[3]{a}$ is also called the cube root.

Be careful where you put the index:
\begin{equation*}
 \sqrt[3]{a} \text{ means the cube root of $a$}
\end{equation*}
But
\begin{equation*}
 3 \sqrt{a} \text{ means } 3 \times \sqrt{a}
\end{equation*}

\Subheading{Odd roots}

If $n$ is \emph{odd}, every real number $a$ has exactly one real $n$-th root $\sqrt[n]{a}$.
It has the same sign as $a$.

\Subheading{Even roots}

If $n$ is \emph{even}, things are more complicated.
\begin{itemize}
\item The equation $z^n = 0$ has only one solution, $z = 0$.
 So $\sqrt[n]{0} = 0$.

\item If $a > 0$, the equation $z^n = a$ has two solutions for $z$.
 The positive solution is $\sqrt[n]{a}$.
 The negative solution is $-\sqrt[n]{a}$.
 The set of two solutions is abbreviated $\pm \sqrt[n]{a}$.

\item If $a < 0$, there is no real solution to $z^n = a$, so for our purposes, $\sqrt[n]{a}$ is undefined.
\end{itemize}

\Subheading{Meaning of a fractional power}

Assuming $\nicefrac{m}{n}$ is in lowest terms,
\begin{equation*}
   a^{\nicefrac{m}{n}}
   = \left(\sqrt[n]{a}\right)^m
   = \sqrt[n]{a^m}
\end{equation*}
This definition only works under certain conditions:
\begin{itemize}
\item If $\frac{m}{n} > 0$:
 \begin{itemize}
 \item If $n$ is odd, this works for all real numbers $a$.
 \item If $n$ is even, this works for $a \geq 0$.
 \end{itemize}
\item If $\frac{m}{n} < 0$:
 \begin{itemize}
 \item If $n$ is odd, this works for $a \neq 0$.
 \item If $n$ is even, this works for $a > 0$.
 \end{itemize}
\end{itemize}

If $m$ and $n$ are both even and positive, there's an additional possibility.
\begin{itemize}
\item This works for all $a$, even if $a < 0$:
 \begin{equation*}
  \sqrt[n]{a^m} = \sqrt[n]{\Abs{a}^m} = \Abs{a}^{\nicefrac{m}{n}}
 \end{equation*}
 However, this can only work for $a \geq 0$:
 \begin{equation*}
  \left(\sqrt[n]{a}\right)^m = a^{\nicefrac{m}{n}}
 \end{equation*}
\item In particular, $\sqrt{a^2} = \Abs{a}$ for every real $a$.
\item But $(\sqrt{a})^2$ is only defined if $a \geq 0$, in which case $(\sqrt{a})^2 = a$.
\end{itemize}

If the exponent $n$ irrational, the power $a^n$ is only defined for $a \geq 0$ and must be defined in terms of logarithms.

\Subheading{Properties of roots}

The following properties hold as long as all operations are defined.

\begin{multicols}{2}

 \begin{FormulaBox}{Convert roots $\longleftrightarrow$ powers}
  \begin{equation*}
   \begin{split}
     \sqrt[n]{a} &= a^{\nicefrac{1}{n}}
     \\
     \frac{1}{\sqrt[n]{a}} &= \frac{1}{a^{\nicefrac{1}{n}}} = a^{-\nicefrac{1}{n}}
     \\
   \end{split}
  \end{equation*}
 \end{FormulaBox}

 \begin{FormulaBox}{Product with same index}
  \begin{equation*}
   \left(\sqrt[n]{\vphantom{b}a}\right)
   \cdot
   \left(\sqrt[n]{b}\right) = \sqrt[n]{a \cdot b}
  \end{equation*}
 \end{FormulaBox}

 \begin{FormulaBox}{Quotient with same index}
  \begin{equation*}
   \frac{\sqrt[n]{a}}{\sqrt[n]{b}}
   =
   \sqrt[n]{\frac{a}{b}}
  \end{equation*}
 \end{FormulaBox}

 \begin{FormulaBox}{Nested roots}
  \begin{equation*}
   \sqrt[m]{\sqrt[n]{a}} = \sqrt[m\cdot n]{a}
  \end{equation*}
 \end{FormulaBox}


\end{multicols}

%%% Local Variables:
%%% mode: latex
%%% TeX-master: "Business-calculus-workbook"
%%% End:
