\Section{Derivative rules---chain rule}

\begin{FormulaBox}{Chain rule, Lagrange notation}
 Given
 \begin{equation*}
  \begin{split}
    \OuterStyle{v(u)} &= \text{ \OuterStyle{function of $u$, \emph{outer function}} }
    \\
    \InnerStyle{u(x)} &= \text{ \InnerStyle{function of $x$, \emph{inner function}} }
    \\
    f(x) &= \OuterStyle{v\bigg(\InnerStyle{u(x)}\bigg)}
    \\
  \end{split}
 \end{equation*}
 Think of $f(x)$ as the composition of $\OuterStyle{v(u)}$ and $\InnerStyle{u(x)}$
 \begin{equation*}
  f'(x) = \OuterStyle{v'\bigg(\InnerStyle{u(x)}\bigg)}
  \cdot
  \InnerStyle{u'(x)}
 \end{equation*}
\end{FormulaBox}

\begin{FormulaBox}{Chain rule, Leibniz notation}
 Given
 \begin{equation*}
  \begin{split}
    y &= \text{ \OuterStyle{function of $u$, \emph{outer function}} }
    \\
    u &= \text{ \InnerStyle{function of $x$, \emph{inner function}} }
    \\
  \end{split}
 \end{equation*}
 Think of $y$ as a function of $x$
 \begin{equation*}
  \Deriv{y}{x} =
  \OuterStyle{\Deriv{y}{u}}
  \cdot
  \InnerStyle{\Deriv{u}{x}}
 \end{equation*}
\end{FormulaBox}

\begin{FormulaBox}{Chain rule, in words}
 \begin{equation*}
  \DerivWord{\Word{composition}}
  =
  \OuterStyle{\DerivWord{\Word{outer}}}\cdot\InnerStyle{\DerivWord{\Word{inner}}}
 \end{equation*}
\end{FormulaBox}

\begin{FormulaBox}{General power rule---combo power and chain}
 \Formula{\D{(\Y)^{\,\P}}}{\P \cdot (\Y)^{\,\P - 1} \cdot \D{\Y}}
\end{FormulaBox}

%%% Local Variables:
%%% mode: latex
%%% TeX-master: "Business-calculus-workbook"
%%% End:
