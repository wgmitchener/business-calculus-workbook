\Section{Derivative practice---abstract functions}

In each of these problems, you will find the derivative of one function that is defined in terms of a second function.
No definition is given for this second function.
The final answer will therefore have to be written using the letter name of the second function or its derivative.
For clarity, Lagrange notation, as in $f(x)$, is used throughout.
A letter followed by an expression in parentheses, as in $f(x)$ or $w(t)$ will always mean function application, not multiplication.
Multiplication is always indicated by a dot $\cdot$ on this problem set.

\begin{ProblemSet}[pencil space=2.5in]

 \begin{Problem}
  What is $f'(x)$?
  Here, $g$ is a function.
  Your answer will have $g(\dots)$ and $g'(\dots)$ in it.
  \begin{equation*}
   f(x) = x^2 \cdot g(x)
  \end{equation*}
 \end{Problem}

 \begin{Problem}[pencil space=3in]
  What is $g'(t)$?
  Here, $w$ is a function.
  Your answer will have $w(\dots)$ and $w'(\dots)$ in it.
  \begin{equation*}
   g(t) = \frac{w(t) + t^2}{w(t) - t^2}
  \end{equation*}
 \end{Problem}

 \begin{Problem}
  What is $f'(x)$?
  Here, $m$ is a function.
  Your answer will have $m'(\dots)$ in it.
  \begin{equation*}
   f(x) = m\!\left(3\cdot x^2 + 7\right)
  \end{equation*}
 \end{Problem}

 \begin{Problem}
  What is $f'(x)$?
  Here, $q$ is a function.
  Your answer will have $q(\dots)$ and $q'(\dots)$ in it.
  \begin{equation*}
   f(x) = (10 + q(3\cdot x))^5
  \end{equation*}
 \end{Problem}

\end{ProblemSet}

%%% Local Variables:
%%% mode: latex
%%% TeX-master: "Business-calculus-workbook"
%%% End:
