\Section{Properties of exponents for real numbers}

In the notation $a^n$, the number $a$ is called the \emph{base}, and the number $n$ is called the \emph{exponent} or \emph{power}.
A whole number exponent $n$ means repeated multiplication:
\begin{equation*}
 a^n \text{ means } \overbrace{a \times a \times \cdots \times a}^{\text{multiply $n$ copies of $a$}}
\end{equation*}
The meaning of a fractional exponent $n$ is given in \cref{sec:roots-frac-exp}.

The following properties hold as long as all operations are defined.

\begin{multicols}{2}

 \begin{FormulaBox}{Easy cases}
  \begin{equation*}
   \begin{split}
     a^0 &= 1  \text{ provided $a \neq 0$ }
     \\
     a^1 &= a
     \\
     a^2 &= a \cdot a
     \\
   \end{split}
  \end{equation*}
 \end{FormulaBox}

 \begin{FormulaBox}{Powers of $1$}
  \begin{equation*}
   1^n = 1
  \end{equation*}
 \end{FormulaBox}

 \begin{FormulaBox}{Powers of $0$}
  \begin{equation*}
   0^n =
   \begin{cases}
     0 & \text{ if $n > 0$}
     \\
     \text{undefined} & \text{ if $n \leq 0$ }
   \end{cases}
  \end{equation*}
 \end{FormulaBox}

 \begin{FormulaBox}{Sign of odd powers}
  If $n$ is \emph{odd}, $a^n$ has the same sign as $a$.
 \end{FormulaBox}

 \begin{FormulaBox}{Sign of even powers}
  If $n$ is \emph{even}, $a^n \geq 0$ no matter what sign $a$ has.
  In this case, $a^n = \Abs{a^n} = \Abs{a}^n$.
 \end{FormulaBox}

 \begin{FormulaBox}{Nested powers}
  \begin{equation*}
   (a^n)^m = a^{n \cdot m}
  \end{equation*}
 \end{FormulaBox}

 \begin{FormulaBox}{Product with same base}
  \begin{equation*}
   a^m \cdot a^n = a^{m+n}
  \end{equation*}
 \end{FormulaBox}

 \begin{FormulaBox}{Product with same exponent}
  \begin{equation*}
   a^n \cdot b^n = (a \cdot b)^n
  \end{equation*}
 \end{FormulaBox}

 \begin{FormulaBox}{Negative exponent}
  As long as $a \neq 0$:
  \begin{equation*}
   \begin{split}
     a^{-1} &= \frac{1}{a}
     \\
     a^{-n} &= \frac{1}{a^n}
     \\
     \frac{c}{a^n} &= c \cdot a^{-n}
     \\
   \end{split}
  \end{equation*}
 \end{FormulaBox}

 \begin{FormulaBox}{Quotient with same base}
  \begin{equation*}
   \frac{a^m}{a^n} = a^m \cdot a^{-n} = a^{m-n}
  \end{equation*}
 \end{FormulaBox}

 \begin{FormulaBox}{Quotient with same exponent}
  As long as $b \neq 0$:
  \begin{equation*}
   \frac{a^n}{b^n} = \pfrac{a}{b}^n
  \end{equation*}
 \end{FormulaBox}

\end{multicols}

%%% Local Variables:
%%% mode: latex
%%% TeX-master: "Business-calculus-workbook"
%%% End:
