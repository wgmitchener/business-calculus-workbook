\Section{Business modeling---core concepts}

There is a standard way to use functions to model a business.
Be sure you understand the conceptual meaning of the symbols used in this setup.

\begin{ProblemSet}
 \begin{Problem}
  In common speech, the terms \emph{cost} and \emph{price} can be used for the same meaning.
  However, in business scenarios, the terms \emph{cost} and \emph{price} have very different meanings.

  In a business context, what does \emph{price} mean?
 \end{Problem}
 \begin{Problem}
  In a business context, what does \emph{cost} mean?
 \end{Problem}
 \begin{Problem}
  In the context of a business scenario, the symbol $x$ usually means what?
  (And you'll have to read carefully:
  Every so often, $x$ gets used for something else, sorry, nothing I can do about it.)

  ``The symbol $x$ means \dots
 \end{Problem}
 \begin{Problem}
  Conceptually, the \emph{demand function} takes as its input a number that means what?
 \end{Problem}
 \begin{Problem}
  Conceptually, the \emph{demand function} produces as its output a number that means what?
 \end{Problem}
 \begin{Problem}
  Conceptually, the \emph{revenue function} takes as its input a number that means what?
 \end{Problem}
 \begin{Problem}
  Conceptually, the \emph{revenue function} produces as its output a number that means what?
 \end{Problem}
 \begin{Problem}
  Conceptually, the \emph{cost function} takes as its input a number that means what?
 \end{Problem}
 \begin{Problem}
  Conceptually, the \emph{cost function} produces as its output a number that means what?
 \end{Problem}
 \begin{Problem}
  Conceptually, the \emph{profit function} takes as its input a number that means what?
 \end{Problem}
 \begin{Problem}
  Conceptually, the \emph{profit function} produces as its output a number that means what?
 \end{Problem}

 \begin{Problem}
  Conceptually, how do you begin building a revenue function?
 \end{Problem}
 \begin{Problem}
  Conceptually, how do you begin building a profit function?
 \end{Problem}
 \begin{Problem}
  What equations do break-even points satisfy?
 \end{Problem}
\end{ProblemSet}

\newpage

Consider the following business scenario.
A jewelry shop sells a particular kind of bracelet.
Each bracelet costs the store $\$15.75$.
The manager estimates that to sell $x$ bracelets per week, the price of a bracelet should be $30.00 - 0.25 x$ dollars.

Answer the following, using equations for final answers whenever possible.
When the result is a number, give a decimal approximation to the nearest cent or hundredth of a unit.
When the result is a function, write an equation giving the function as a simplified, explicit expression, in terms of $x$, that does not reference other functions.

\begin{ProblemSet}
 \begin{Problem}
  Conceptually, what does the symbol $x$ represent?
 \end{Problem}
 \begin{Problem}
  Which function (demand, revenue, cost, profit), is given by the expression $30.00 - 0.25 x$?
 \end{Problem}
 \begin{Problem}
  The number $\$15.75$ is directly relevant to which function (demand, revenue, cost, profit)?
 \end{Problem}
 \begin{Problem}
  Build a formula for the cost function, $C(x) = \dots$
 \end{Problem}
 \begin{Problem}
  Build a formula for the demand function, $D(x) = \dots$
 \end{Problem}
 \begin{Problem}
  Build a formula for the revenue function, $R(x) = \dots$
 \end{Problem}
 \begin{Problem}
  Build a formula for the profit function, $P(x) = \dots$
 \end{Problem}
 \begin{Problem}
  How much does it cost the store to acquire $20$ bracelets?
 \end{Problem}
 \begin{Problem}
  If the store plans to sell $20$ bracelets per week, at what price should they be sold?
 \end{Problem}
 \begin{Problem}
  If the store sells $20$ bracelets per week, what is the weekly revenue?
 \end{Problem}
 \begin{Problem}
  If the store sells $20$ bracelets per week, what is the weekly profit?
 \end{Problem}
 \begin{Problem}
  If the store plans to sell bracelets for $\$22.00$, how many should they plan to sell on average each week?
  (It's okay for this to be a fractional amount.)
 \end{Problem}
 \begin{Problem}
  What are the two break-even points for this business?
 \end{Problem}
\end{ProblemSet}

\newpage

Here's another business scenario.

A factory produces screwdrivers. Suppose it costs
\begin{equation*}
 10.00 + 1.75 x \text{ dollars}
\end{equation*}
to produce $x$ screwdrivers.
If the price of a screwdriver is $\$8$, the manufacturer can sell $300$ of them per month.
However, they can only sell $250$ of them per month if the price is $\$10$.

Fully simplify all answers.
Give answers as equations using proper notation when possible.
Give decimal approximations to the nearest cent or hundredth of a unit.

\begin{ProblemSet}
 \begin{Problem}[pencil space=0.5in]
  On the following grid, set up axes that are appropriate for sketching the graph of the demand function.
  Label each axis with the concept that it represents.
  Plot two points based on the given information, and draw the line through these points.
  This line is the graph of the demand function.
  \vspace{0.5in}

  \GraphingGridSmall

 \end{Problem}

 \begin{Problem}
  Assuming the demand function is linear, build a formula for it.
  Write it as an equation $D(x) = \dots$
 \end{Problem}

 \begin{Problem}
  What is the cost function?
 \end{Problem}

 \begin{Problem}
  What is the revenue function?
 \end{Problem}

 \begin{Problem}
  What is the profit function?
 \end{Problem}

 \begin{Problem}
  At what price should screwdrivers be sold to sell $220$ per month?
 \end{Problem}

 \begin{Problem}
  How many will be sold if the price is $\$11.00$?
 \end{Problem}

 \begin{Problem}
  How much does it cost to produce $250$ screwdrivers per month?
 \end{Problem}

 \begin{Problem}
  What is the profit if $250$ screwdrivers are sold?
 \end{Problem}

 \begin{Problem}
  What are the break-even points for this business?
 \end{Problem}

\end{ProblemSet}

%%% Local Variables:
%%% mode: latex
%%% TeX-master: "Business-calculus-workbook"
%%% End:
