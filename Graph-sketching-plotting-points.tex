\Section{Graph sketching---plotting points}

The graph of a function $f$ is a set of points in a coordinate plane.
We usually associate the horizontal axis with an independent variable, typically $x$, and the vertical axis with a dependent variable, typically $y$, and label the graph $y = f(x)$.
Each number $x$ in the domain of $f$ corresponds to a point $(x, f(x))$.

Consider the function
\begin{equation*}
 f(x) = -\frac{1}{5} x^2 (x - 4)
\end{equation*}
\begin{ProblemSet}[pencil space=0.5in]
 \begin{Problem}[pencil space=0in]
  Find the coordinates of the points on the graph of $f$ whose $x$-coordinates are $-2$, $-1$, $0$, $1$, $2$, $3$, $4$, $5$.
  List them in this table.
  Round the $y$-coordinates to the nearest tenth.

  \begin{tikzpicture}
   \path[use as bounding box] (-1,-1) rectangle (9,10);
   \draw[color=GraphingGridColor, line width=\GraphingGridLineWidth,
   ]
   (0, 0) grid[xstep=4,ystep=1] (8, 8)
   ;
   \node at (2,8.5) {$x$};
   \node at (6,8.5) {$f(x)$};
  \end{tikzpicture}
 \end{Problem}
 \begin{Problem}
  Choose an $x$-range, that is, a number $\xMin$ a bit less than all those $x$-values and a number $\xMax$ a bit greater than all those $x$-values.
 \end{Problem}
 \begin{Problem}
  Choose a $y$-range, that is, a number $\yMin$ a bit less than all those $y$-values and a number $\yMax$ a bit greater than all those $y$-values.
 \end{Problem}
 \begin{Problem}
  Draw axes on the grid below and set up a coordinate system that includes the interval from $\xMin$ to $\xMax$ horizontally, and $\yMin$ to $\yMax$ vertically.
  Draw tick marks and numbers on the axes to indicate the scale.
  The horizontal and vertical scales don't have to be the same and you don't have to use the whole grid.
  Then plot the points you wrote down in the table and draw a smooth curve connecting them.
  Use a calculator to check your picture.

  \bigskip
  \GraphingGrid
 \end{Problem}
 \begin{Problem}
  Use the graph to estimate $f(1.5)$ and check it with your calculator.
 \end{Problem}
 \begin{Problem}
  Use the graph to estimate the three solutions to $f(x) = 1$ and check them with your calculator.
 \end{Problem}
\end{ProblemSet}

%%% Local Variables:
%%% mode: latex
%%% TeX-master: "Business-calculus-workbook"
%%% End:
