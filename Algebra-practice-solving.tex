\Section{Algebra practice---solving}

Solve the following equations for the unknown variable.
You may need to do some simplifications as you solve.
Expect to use lots of inverse operations.
(This is sometimes called ``peeling the onion.'')
Your final answer should be an integer or a rational number.
Improper fractions, as in $\nicefrac{5}{2}$, are okay.

\begin{multicols}{2}
 \begin{ProblemSet}[pencil space=3.5in]
  \EqProb{6x + 2 = 4x - 1}
  \EqProb{\frac{2 + 2q}{3} = \frac{1}{2} q}

  \EqProb{\frac{1}{x} + \frac{1}{2} = 3}

  \EqProb{5 = \sqrt{16 - x}}
 \end{ProblemSet}
\end{multicols}

\newpage

Solve these and give decimal approximations to the nearest thousandth, as in $0.123$ or $1.234$.

\begin{ProblemSet}[pencil space=3.5in]
 \EqProb{3.20 x + 0.75 = 122.85}
 \EqProb{\frac{3.38}{2.2 t - 0.5} = 0.09}
\end{ProblemSet}


\newpage
\Subheading{Extra practice}

Solve these in the same way.

\begin{multicols}{2}
 \begin{ProblemSet}[pencil space=0in]
  \EqProb{-3 t + 4 = 7 - t}
  \EqProb{3r + \frac{1}{2} = \frac{2r}{5} - 2}

  \EqProb{\frac{2}{3 - t} = \frac{5}{4}}
  \EqProb{\frac{b}{1 + b} = 2}
  \EqProb{-\frac{2}{5} = 1 + \frac{4}{3u}}

  \EqProb{3 \sqrt{9 + w} = 7}
  \EqProb{3 \sqrt{v} - 5 = 1}
  \EqProb{\frac{3}{\sqrt{q}} - 5 = 1}
 \end{ProblemSet}
\end{multicols}

%%% Local Variables:
%%% mode: latex
%%% TeX-master: "Business-calculus-workbook"
%%% End:
