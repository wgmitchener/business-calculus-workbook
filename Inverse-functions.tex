\Section{Inverse functions}

Given a function $f$, recall that the inverse function $f^{-1}$ is defined by
\begin{equation*}
 f^{-1}(y) = \text{ the $x$ that solves } f(x) = y
\end{equation*}
This is a special use of the superscript $-1$ and it \emph{does not} mean an ordinary power of $-1$.
Note that the inverse function for $f^{-1}$ is the original $f$.

\begin{ProblemSet}
 \begin{Problem}
  What is the inverse of the function $f(x) = x^3$?
 \end{Problem}
 \begin{Problem}
  What is the inverse of the function
  \begin{LeftEquation}
   f(x) = \frac{1}{x^{\nicefrac{1}{5}}}?
  \end{LeftEquation}
 \end{Problem}
\end{ProblemSet}

Inverse functions can be used to solve equations, because
$f$ and $f^{-1}$ undo each other:
\begin{equation*}
 f^{-1}\big( f(z) \big) = z
 \text{ and }
 f\big( f^{-1}(w) \big) = w
\end{equation*}

Solve these equations for $x$.
Assume that $f$ is some function whose inverse is $f^{-1}$.
Your answers might refer to $f$ or $f^{-1}$.
\begin{ProblemSet}[pencil space=2.5in]
 \EqProb{f(4 + x) = 5}
 \EqProb{10 - \frac{4}{f(2 x)} = 3}
 \EqProb{2 \cdot f^{-1}(3 - x) = 8}
 \EqProb{-6 = \frac{2}{f^{-1}(4 x) - 3}}
\end{ProblemSet}

%%% Local Variables:
%%% mode: latex
%%% TeX-master: "Business-calculus-workbook"
%%% End:
