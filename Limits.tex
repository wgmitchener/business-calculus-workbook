\Section{Limits}

Consider the function
% ( (x - 5)(x + 5) ) / ( (x - 5)(x - 7) )
% (x^2 - 25) / (x^2 - 12x + 35)
\begin{equation*}
 f(x) = \frac{x^2 - 25}{x^2 - 12x + 35}
\end{equation*}
Enter $f$ as a function in your calculator.

\begin{ProblemSet}
 \begin{Problem}
  What is the domain of $f$?
  Hint: It should be all real numbers except for two numbers that must be excluded because they would result in division by zero.
 \end{Problem}
 \begin{Problem}[pencil space=2in]
  Pick one of the gaps in the domain of $f$.
  Use a calculator to evaluate $f$ near that value of $x$.
  (For example, if there's a gap at $x = 2$, then $f(2)$ is undefined, so look at $f(1.9)$, $f(1.99)$, \dots and $f(2.1)$, $f(2.01)$, \dots)
  Write down a table of $x$ and $y$ values.
  Does it look like $f$ approaches a limit at that value of $x$?
  If so, write an equation using limit notation for what you found.
  Otherwise, write that the limit does not exist.
 \end{Problem}
 \begin{Problem}[pencil space=2in]
  Do the same for the other gap.
 \end{Problem}
 \begin{Problem}
  Choose an $x$-range, that is, minimum and maximum values of $x$ that include the gaps and some space to the left and right.
  Write it here.
 \end{Problem}
 \begin{Problem}
  Use a calculator to sketch the graph of $f$ using the $x$-range you just found.
  Select a $y$-range that lets you see all the interesting features of $f$.
  Sketch the graph on the grid below.
  Label the axes.
  Draw a dotted line $\vdots$ for the vertical asymptote.
  Draw a small circle $\circ$ for the missing point.
  \bigskip

  \GraphingGrid
 \end{Problem}
 \begin{Problem}[pencil space=3in]
  Now you'll use an algebraic method to confirm the numerical results.
  Factor the numerator and denominator of $f$.
  Cancel common factors.
  Write an equation defining $f^*(x) =$ this simplified expression.
 \end{Problem}
 \begin{Problem}
  What is the domain of $f^*$?
  Hint: It should have only one gap.
 \end{Problem}
 \begin{Problem}[pencil space=2in]
  The domain of $f^*$ includes one value of $x$ that is not in the domain of $f$, so the limit of $f$ exists at that number, even though $f$ is not defined at that number.
  Use $f^*$ to compute the value of the limit.
  Express what you found using limit notation.
 \end{Problem}
\end{ProblemSet}

%%% Local Variables:
%%% mode: latex
%%% TeX-master: "Business-calculus-workbook"
%%% End:
