\Section{Algebra practice---function notation}

Let's define the following functions using Lagrange notation:
\begin{equation*}
 \begin{split}
   f(x) &= 3x - 4
   \\
   g(x) &= 2\cdot(x + 1)
   \\
   h(x) &= x^2 + 1
 \end{split}
\end{equation*}
Letters other than $f$, $g$, and $h$ are real-valued variables.

For this collection of problems, substitute using the definitions of the functions and simplify the result to an equation $z = $ a number or a polynomial in expanded form.

\begin{multicols}{2}
 \begin{ProblemSet}[pencil space=2in]
  \EqProb{z = f(4)}
  \EqProb{z = 3 g(-3) + 2 h(1)}
  \EqProb{z = f(h(2) + 3)}
  \EqProb{z = f(g(2 + 3))}
  \EqProb{z = f(a + 4)}
  \EqProb{z = g(x + 4)}
 \end{ProblemSet}
\end{multicols}

\newpage

Continuing with the same definitions for $f$, $g$, and $h$, answer the following.

\begin{ProblemSet}[pencil space=2.75in]
 \begin{Problem}
  What is the solution to $f(x) = 4$?
 \end{Problem}
 \begin{Problem}
  What is the solution to $f(x + 3) = 4$?
 \end{Problem}
 \begin{Problem}
  What is the solution to $g(2x) + 3 = 5$?
 \end{Problem}
% \begin{Problem}
%  What are the solutions to $h(x) = 10$?
% \end{Problem}
\end{ProblemSet}

\newpage

Recall that the \emph{inverse} of a function $f$ is another function $f^{-1}$ (where the superscript $-1$ is special notation, not an exponent, sorry for the confusion, nothing I can do about it).
The definition of $f^{-1}$ is that given a number $y$, $f^{-1}(y)$ is the number $x$ that solves $f(x) = y$.

Continuing with the same definitions for $f$ and $g$, answer the following.

\begin{ProblemSet}[pencil space=3.5in]
 \begin{Problem}
  Find an explicit expression for $f^{-1}$
 \end{Problem}
 \begin{Problem}
  Find an explicit expression for $g^{-1}$
 \end{Problem}
\end{ProblemSet}


\newpage
\Subheading{Extra practice}

Continuing with the same definitions for $f$, $g$, and $h$, substitute and simplify:

\begin{multicols}{2}
 \begin{ProblemSet}[pencil space=0in]
  \EqProb{z = f(g(2)) + 3}
  \EqProb{z = g(h(-2))}
  \EqProb{z = g(-h(2))}
  \EqProb{z = h(g(-2))}
  \EqProb{z = -g(h(2))}
  \EqProb{z = f(x + 4)}
  \EqProb{z = g(a + 4)}
  \EqProb{z = h(x - 2)}
  \EqProb{z = h(x + 4) + h(x - 1)}
  \EqProb{z = f(2x)^2}
 \end{ProblemSet}
\end{multicols}

%%% Local Variables:
%%% mode: latex
%%% TeX-master: "Business-calculus-workbook"
%%% End:
