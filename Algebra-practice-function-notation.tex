\Section{Algebra practice---function notation}

\Subheading{Lagrange notation}

Let's define the following functions using Lagrange notation:
\begin{equation*}
 \begin{split}
   f(x) &= 3x - 4
   \\
   g(x) &= 2\cdot(x + 1)
   \\
   h(x) &= x^2 + 1
 \end{split}
\end{equation*}
Letters other than $f$, $g$, and $h$ are real-valued variables.

For this collection of problems, substitute using the definitions of the functions and simplify the result to an equation $z = $ a number or a polynomial in expanded form.

\begin{multicols}{2}
 \begin{ProblemSet}[pencil space=1.75in]
  \EqProb{z = f(4)}
  \EqProb{z = 3 g(-3) + 2 h(1)}
  \EqProb{z = f(h(2) + 3)}
  \EqProb{z = f(g(2 + 3))}
  \EqProb{z = f(a + 4)}
  \EqProb{z = g(x + 4)}
 \end{ProblemSet}
\end{multicols}

\newpage

Continuing with the same definitions for $f$, $g$, and $h$, answer the following.

\begin{ProblemSet}[continue,pencil space=2.75in]
 \begin{Problem}
  What is the solution to $f(x) = 4$?
 \end{Problem}
 \begin{Problem}
  What is the solution to $f(x + 3) = 4$?
 \end{Problem}
 \begin{Problem}
  What is the solution to $g(2x) + 3 = 5$?
 \end{Problem}
% \begin{Problem}
%  What are the solutions to $h(x) = 10$?
% \end{Problem}
\end{ProblemSet}

\newpage

Recall that the \emph{inverse} of a function $f$ is another function $f^{-1}$ (where the superscript $-1$ is special notation, not an exponent, sorry for the confusion, nothing I can do about it).
The definition of $f^{-1}$ is that given a number $y$, $f^{-1}(y)$ is the number $x$ that solves $f(x) = y$.

Continuing with the same definitions for $f$ and $g$, answer the following.

\begin{ProblemSet}[continue,pencil space=3.5in]
 \begin{Problem}
  Find an explicit expression for $f^{-1}$
 \end{Problem}
 \begin{Problem}
  Find an explicit expression for $g^{-1}$
 \end{Problem}
\end{ProblemSet}


\newpage
\Subheading{Extra practice}

Continuing with the same definitions for $f$, $g$, and $h$, substitute and simplify:

\begin{multicols}{2}
 \begin{ProblemSet}[continue,pencil space=0in]
  \EqProb{z = f(g(2)) + 3}
  \EqProb{z = g(h(-2))}
  \EqProb{z = g(-h(2))}
  \EqProb{z = h(g(-2))}
  \EqProb{z = -g(h(2))}
  \EqProb{z = f(x + 4)}
  \EqProb{z = g(a + 4)}
  \EqProb{z = h(x - 2)}
  \EqProb{z = h(x + 4) + h(x - 1)}
  \EqProb{z = f(2x)^2}
 \end{ProblemSet}
\end{multicols}

\newpage

\Subheading{Leibniz notation}

In Leibniz notation, a dependent variable is given as a function of an independent variable by a defining equation.
The vertical bar notation is used to express ``plugging in.''

For example, we could define $f(x)$ using Lagrange notation:
\begin{equation*}
 f(x) = 5 x - 4
\end{equation*}
or define $y$ as a function of $x$ using Leibniz notation:
\begin{equation*}
 y = 5 x - 4
\end{equation*}
To plug in 10 for $x$ using Lagrange notation:
\begin{equation*}
 \begin{split}
   f(10) &= 5 \cdot 10 - 4
   \\
   & \text{ which simplifies to }
   \\
   f(10) &= 46
 \end{split}
\end{equation*}
To plug in 10 for $x$ using Leibniz notation:
\begin{equation*}
 \begin{split}
   \Where{y}{x=10} &= 5 \cdot 10 - 4
   \\
   & \text{ which simplifies to }
   \\
   \Where{y}{x=10} &= 46
 \end{split}
\end{equation*}
To plug in $a + b$ for $x$:
\begin{equation*}
 \begin{split}
   \text{Lagrange: } f(a + b) &= 5 \cdot (a + b) - 4
   \\
   \text{Leibniz: } \Where{y}{x=a+b} &= 5 \cdot (a + b) - 4
 \end{split}
\end{equation*}

Evaluate these for some practice with Leibniz notation:

\begin{multicols}{2}
 \begin{ProblemSet}[pencil space=1in]
  \begin{Problem}
   Given
   \begin{equation*}
    y = 4 + 3 x
   \end{equation*}
   find
   \begin{equation*}
    \Where{y}{x = 10}
   \end{equation*}
  \end{Problem}
  \begin{Problem}
   Given
   \begin{equation*}
    y = x^2 - 5 x
   \end{equation*}
   find
   \begin{equation*}
    \Where{y}{x = -2}
   \end{equation*}
  \end{Problem}
 \end{ProblemSet}
\end{multicols}

\newpage

We also use the vertical bar notation for other kinds of ``plugging in''.
For example:
\begin{equation*}
 \Where{5u^{-1} - 2v}{u = 2, v = 3} = 5 \cdot 2^{-1} - 2 \cdot 3
\end{equation*}
\begin{equation*}
 \Where{5u^{-1} - 2v}{u = 3p, v = 1-q} = 5 \cdot (3p)^{-1} - 2 \cdot (1-q)
\end{equation*}
Give these a try:

\begin{multicols}{2}
 \large
 \begin{ProblemSet}[pencil space=1.5in]
  \EqProb{ \Where{4 x - 10}{x = 3} }
  \EqProb{ \Where{r^2 + \frac{1}{r}}{r = k + 1} }
  \EqProb{ \Where{3 x^2 + 8 x - 4 w}{x = 2, w = -1} }
  \EqProb{ \Where{\frac{2 - a}{3 + b}}{a = x+2, b=x^2} }
  \EqProb{ \Where{\big(\Where{4 u - 10}{u = 2t - 1}\big)}{t = 1} }
  \EqProb{ \Where{x + x^2}{x = x + 1} }
 \end{ProblemSet}
\end{multicols}

\newpage

Sometimes there are chains of dependent variables.
For example, suppose we define these variables:
\begin{equation*}
 \begin{split}
   y &= 3 u + 2 \\
   u &= 6 x - 1
 \end{split}
\end{equation*}
Then the notation $\Where{y}{x = 2}$ means to set $x = 2$ and let that information flow through the definitions of the dependent variables.
In this case, it means to plug in 2 for $x$ and evaluate $u$, then substitute that value for $u$ in the definition of $y$.
We start here:
\begin{equation*}
 \begin{split}
   \Where{u}{x=2} &= 6 \cdot 2 - 1
   \\
   \Where{u}{x=2} &= 11
 \end{split}
\end{equation*}
Then use that result:
\begin{equation*}
 \begin{split}
   \Where{y}{x=2} &= \Where{y}{u=11}
   \\
   \Where{y}{x=2} &= \Where{3 u + 2}{u=11}
   \\
   \Where{y}{x=2} &= 3 \cdot 11 + 2
   \\
   \Where{y}{x=2} &= 35
 \end{split}
\end{equation*}
We could also evaluate this way:
\begin{equation*}
 \begin{split}
   \Where{y}{x=2} &= \Where{3 u + 2}{x=2} \text{ using the definition for $y$ as a function of $u$ }
   \\
   \Where{y}{x=2} &= \Where{3 (6x - 1) + 2}{x=2} \text{ using the definition for $u$ as a function of $x$ }
   \\
   \Where{y}{x=2} &= 3 (6 \cdot 2 - 1) + 2
   \\
   \Where{y}{x=2} &= 35
 \end{split}
\end{equation*}

\newpage

Let's define these variables:
\begin{equation*}
 \begin{split}
   y &= 3 - 7 u \\
   u &= 2 + x^2
 \end{split}
\end{equation*}
Evaluate these:

\begin{multicols}{2}
 \begin{ProblemSet}
  \EqProb{\Where{y}{u = 4}}
  \EqProb{\Where{u}{x = 3}}
  \EqProb{\Where{y}{x = 3}}
  \EqProb{\Where{y}{x = 5}}
  \EqProb{\Where{x + 2y}{x = 5}}
  \EqProb{\Where{3u + 2y - x^2}{x = 3}}
  \EqProb{\Where{\left(\Where{y}{x = 2t}\right)}{t = -1}}
  \EqProb{\Where{u - 2 x^2 + y}{x = x + 3}}
 \end{ProblemSet}
\end{multicols}


%%% Local Variables:
%%% mode: latex
%%% TeX-master: "Business-calculus-workbook"
%%% End:
