\Section{Business modeling---marginal analysis}
\label{sec:biz-mod-marginal}

Recall that in a business scenario, \emph{marginal [thing]} is the derivative of [thing] with respect to the number of units of the product that are produced and sold.
Marginals approximate the change that would result if the business made an incremental change, that is, supposing it produced and sold one more unit of its product.

% For example, marginal cost is the derivative of the cost function with respect to the number of units produced.
% The notation for the marginal cost function is $C'(x)$.
% The exact change in the business's costs if it made that incremental change is $C(x+1) - C(x)$, and $C'(x) \approx C(x+1) - C(x)$.
% Marginal cost is approximately how much it would cost to produce and sell one additional unit after producing and selling $x$ units.

% It has to be a function of $x$:
% If our business has been selling only few units (small $x$), producing and selling one more might be inexpensive (small $C'(x)$).
% However, if our business has been selling many units (large $x$), producing and selling an additional unit might result in surprising new expenses, such the need to buy additional components from more expensive sources, paying workers overtime, and paying for more storage space (large $C'(x)$).

% Marginal revenue is $R'(x)$, and marginal profit is $P'(x)$.

Consider the following business scenario.
A jewelry shop sells a particular kind of necklace.
Each week, the shop can order $x$ of these necklaces from a wholesaler for $12.25 + 8.75 x$ dollars.
The manager estimates that to sell $x$ necklaces per week, the price of a necklace should be $25.00 - 0.25 x$ dollars.

Answer the following, using equations for final answers whenever possible.
When the result is a number, give a decimal approximation to the nearest cent or hundredth of a unit.
When the result is a function, write an equation giving the function as a simplified, explicit expression, in terms of $x$, that does not reference other functions.

\begin{ProblemSet}

 \begin{Problem}
  What is the demand function?
 \end{Problem}

 \begin{Problem}
  What is the cost function?
 \end{Problem}

 \begin{Problem}
  What is the revenue function?
 \end{Problem}

 \begin{Problem}
  What is the profit function?
 \end{Problem}

 \begin{Problem}
  What is the marginal cost function?
 \end{Problem}

 \begin{Problem}
  What is the marginal revenue function?
 \end{Problem}

 \begin{Problem}
  What is the marginal profit function?
 \end{Problem}

 \begin{Problem}[pencil space=4in]
  At what values of $x$ are the break-even points for this business?
 \end{Problem}

 \begin{Problem}
  What profit does the shop earn if they sell exactly enough necklaces to break even?
 \end{Problem}

 \begin{Problem}
  Suppose the shop sells exactly enough necklaces to break even.
  At what price will necklaces be sold to accomplish this?
  Give your answer to the nearest cent.
  (There are two prices because there are two break-even points.)
 \end{Problem}

\end{ProblemSet}

%%% Local Variables:
%%% mode: latex
%%% TeX-master: "Business-calculus-workbook"
%%% End:
