\Section{Warnings and common mistakes}

\begin{WarningBox}{Negation is like multiplication: $-1\times \dots$}
 A negative sign before a number is like multiplication,
 so apply it \emph{after} an exponent on that number.
 \begin{equation*}
  -3^2 = -1 \times (3^2) = -9
 \end{equation*}
 but
 \begin{equation*}
  (-3)^2 = (-1 \times 3)^2 = 9
 \end{equation*}
\end{WarningBox}
\begin{WarningBox}{Substitutions must respect structure}
 Sometimes you need extra parentheses:
 \begin{equation*}
  \Where{u^2}{u = a + b} = \left(a + b\right)^2 \qquad \text{ \emph{and not} } a + b^2
 \end{equation*}
 In the quadratic formula, for example
 \begin{equation*}
  \Where{b^2}{b = -3} = (-3)^2 = 9 \qquad \text{ \emph{and not} } -3^2 = -9
 \end{equation*}
\end{WarningBox}
\begin{WarningBox}{Fractions can be split on $+$ or $-$ only in the numerator}
 Splitting the numerator is okay:
 \begin{equation*}
  \frac{2 + x}{7} = \frac{2}{7} + \frac{x}{7}
 \end{equation*}
 But splitting the denominator \emph{does not work:}
 \begin{equation*}
  \frac{7}{2 + x} \qquad \text{ \emph{is not} } \frac{7}{2} + \frac{7}{x}
 \end{equation*}
\end{WarningBox}
\begin{WarningBox}{Fractions can be spread out on multiplication}
 When you have multiplication of factors divided by multiplication of factors,
 you can spread out the fraction:
 \begin{equation*}
  \frac{x^2 y}{x^2 (3 + z)}
  = \frac{x^2}{x^2} \cdot \frac{y}{1} \cdot \frac{1}{3 + z}
 \end{equation*}
 But when you have addition or subtraction as the outermost operation in the numerator or denominator, spreading out \emph{does not work:}
 \begin{equation*}
  \frac{x^2 + y}{x^2 (3 + z)}
  \qquad \text{ \emph{is not} }
  \frac{x^2}{x^2} + \frac{y}{1} + \frac{1}{3 + z}
 \end{equation*}
\end{WarningBox}
\begin{WarningBox}{Cancel only factors}
 When you have multiplication of factors divided by multiplication of factors,
 canceling a common factor is an option, as long as it's nonzero:
 \begin{equation*}
  \frac{x^2 y}{x^2 (3 + z)}
  = \frac{\cancel{x^2} y}{\cancel{x^2} (3 + z)}
  = \frac{y}{3 + z}
 \end{equation*}
 But when you have addition or subtraction as the outermost operation in the numerator or denominator, canceling \emph{does not work:}
 \begin{equation*}
  \frac{x^2 + y}{x^2 (3 + z)}
  \qquad \text{ \emph{is not} } \frac{y}{3 + z}
 \end{equation*}
 \begin{equation*}
  \frac{x^2 y}{x^2 + (3 + z)}
  \qquad \text{ \emph{is not} } \frac{y}{3 + z}
 \end{equation*}
 \begin{equation*}
  \frac{x^2 + y}{x^2 + 3 + z}
  \qquad \text{ \emph{is not} } \frac{y}{3 + z}
 \end{equation*}
\end{WarningBox}
\begin{WarningBox}{Power applied to $+$ or $-$ requires FOIL}
 When a power is applied to \emph{multiplication}, you can distribute the power:
 \begin{equation*}
  (5 x)^2 = 5^2 x^2 = 25 x^2
 \end{equation*}
 But when a power is applied to \emph{addition or subtraction}, you \emph{can't} distribute the power.
 You do have the option of expanding with FOIL:
 \begin{equation*}
  (5 + x)^2 = (5 + x)(5 + x) = 25 + 10x + x^2
  \qquad \text{ \emph{which is not} } 25 + x^2
 \end{equation*}
\end{WarningBox}
\begin{WarningBox}{Root applied to $+$ or $-$ is mostly hopeless}
 When a root is applied to \emph{multiplication}, you can distribute the root:
 \begin{equation*}
  \sqrt{9 x} = \sqrt{9} \cdot \sqrt{x} = 3 \sqrt{x}
 \end{equation*}
 But when a root is applied to \emph{addition or subtraction}, you \emph{can't} distribute:
 \begin{equation*}
  \sqrt{9 - x} \qquad \text{ \emph{is not} } 3 - \sqrt{x}
 \end{equation*}
 and there's usually not much else you can do with that.
\end{WarningBox}
\begin{WarningBox}{When solving, apply operations to all of both sides}
 You can multiply all of both sides of an equation by a number:
 \begin{equation*}
  \begin{split}
    \frac{3 + x}{2}
    &= 1 + x
    \\
    2 \cdot \left(\frac{3 + x}{2}\right)
    &= \left(1 + x\right) \cdot 2
      \qquad \text{ \emph{which is not} } 1 + x \cdot 2
    \\
    \cancel{2} \cdot \left(\frac{3 + x}{\cancel{2}}\right)
    &= 2 + 2x
      \qquad \text{ \emph{which is not} } 1 + 2x
    \\
    3 + x &= 2 + 2x
  \end{split}
 \end{equation*}
\end{WarningBox}
\begin{WarningBox}{Multiplying a fraction is not the same as changing terms}
 Multiply by $1$ to change terms:
 \begin{equation*}
  \begin{split}
    \frac{x}{2} &= 1 \cdot \frac{x}{2}
    \\
    \frac{x}{2} &= \frac{3}{3} \cdot \frac{x}{2}
    \\
    \frac{x}{2} &= \frac{3 \cdot x}{3 \cdot 2}
    \\
    \frac{x}{2} &= \frac{3x}{6}
  \end{split}
 \end{equation*}
 But multiplying a fraction by a number works like this:
 \begin{equation*}
  \begin{split}
    3 \cdot \frac{x}{2} &= \frac{3}{1} \cdot \frac{x}{2}
      \qquad \text{ \emph{which is not} } \frac{3 \cdot x}{3 \cdot 2}
    \\
    3 \cdot \frac{x}{2} &= \frac{3 \cdot x}{1 \cdot 2}
    \\
    3 \cdot \frac{x}{2} &= \frac{3x}{2}
  \end{split}
 \end{equation*}
\end{WarningBox}
\begin{WarningBox}{Careful with calculators and denominators}
 When using left-to-right notation, calculators group division and multiplication together, so
 \begin{equation*}
  \mathtt{60 / 3 * 4}
  \text{ means }
  \left(\nicefrac{60}{3}\right) \cdot 4
  = 80
 \end{equation*}
 If your denominator includes multiplication, you need extra parentheses
 \begin{equation*}
  \text{ To compute }
  \frac{60}{3 \cdot 4}
  \text{ enter }
  \mathtt{60 / (3 * 4)}
 \end{equation*}
\end{WarningBox}

%%% Local Variables:
%%% mode: latex
%%% TeX-master: "Business-calculus-workbook"
%%% End:
