\Section{Logarithm practice}

\textbf{Take-apart problems:}
Re-write these expressions so that $\ln$ is no longer applied directly to multiplication, division, or a power.

\begin{multicols}{2}
 \begin{ProblemSet}[pencil space=3.5in]
  \EqProb{z = \LN{x^2\cdot y^{-3}}}
  \EqProb{z = \LN{\frac{\sqrt{5x}}{3 y^2}}}
  \EqProb{z = \LN{\pfrac{y}{7 x^2}^{3}}}
  \EqProb{z = \LN{\left(x + 3\right)^{\nicefrac{2}{3}} \cdot \sqrt{5 - y}}}
 \end{ProblemSet}
\end{multicols}
\newpage

\textbf{Put-together problems:}
Re-write these expressions as $z = \LN{\dots}$, that is, everything inside one application of $\ln$.

\begin{multicols}{2}
 \begin{ProblemSet}[pencil space=4in]
  \EqProb{z = 3 \LN{x} + 4 \LN{y}}
  \EqProb{z = 2 \LN{y} - \frac{1}{2} \cdot \LN{x}}
  \EqProb{z = \LN{x+5} + \LN{x-1}}
  \EqProb{z = \frac{3}{5} \cdot \bigg(\LN{y^2} - \LN{x}\bigg)}
 \end{ProblemSet}
\end{multicols}
\newpage

\textbf{Equation solving:}
Solve the following equations.
For some of these, it will help if you put the equation in a factored form and remember this:
\begin{quote}
 The solutions to an equation of the form $u(x) \cdot v(x) = 0$ are the solutions to $u(x) = 0$ together with the solutions to $v(x) = 0$.
\end{quote}

\begin{multicols}{2}
 \begin{ProblemSet}[pencil space=4in]
  \EqProb{4^x = 65}
  \EqProb{10 - 3 \LN{x + 2} = 4}
  \EqProb{\LN{4x} + \LN{3x} = 5}
  \EqProb{5 \me^{-0.3 t} = 60.3}

  \EqProb{(6x - 2) \cdot\LN{2x} = 0}
  \EqProb{\me^{-2x}(x^2 - 9) = 0}
  \EqProb{x^2 \me^{-x} + x \me^{-x} = 0}
  \EqProb{x^2 \me^{x} = 4 x^2}
 \end{ProblemSet}
\end{multicols}

\newpage

Compound interest problems.

\begin{ProblemSet}[pencil space=4in]
 \begin{Problem}
  I have \$3000 to invest.
  I would like to have \$3300 in two years.
  What annual percentage rate is required if interest is compounded continuously?
 \end{Problem}

 \begin{Problem}
  I deposit \$4000 in an account that earns interest at an APR of $7\%$ compounded continuously.
  How long will it take for the account value to double?
 \end{Problem}
\end{ProblemSet}

%%% Local Variables:
%%% mode: latex
%%% TeX-master: "Business-calculus-workbook"
%%% End:
