\Section{Antiderivative rules---basics}

In these formulas, the $\cdots + C$ is the constant of integration.

\begin{multicols}{2}

 \begin{FormulaBox}{Antiderivative of 0}
  \begin{equation*}
   \IndefiniteIntegral{ 0 }{x} = C
  \end{equation*}
 \end{FormulaBox}

 \begin{FormulaBox}{Antiderivative of 1}
  \begin{equation*}
   \begin{split}
     \IndefiniteIntegral{}{x} &= x + C
     \\
     \IndefiniteIntegral{ 1 }{x} &= x + C
   \end{split}
  \end{equation*}
 \end{FormulaBox}

 \begin{FormulaBox}{Antiderivative of constant factor}
  \begin{equation*}
   \IndefiniteIntegral{ A \, f(x) }{x}
   =
   A \IndefiniteIntegral{ f(x) }{x}
  \end{equation*}
 \end{FormulaBox}

 \begin{FormulaBox}{Antiderivative of addition}
  \begin{equation*}
   \IndefiniteIntegral{ \big( f(x) + g(x) \big) }{x}
   =
   \IndefiniteIntegral{ f(x) }{x}
   + \IndefiniteIntegral{ g(x) }{x}
  \end{equation*}
 \end{FormulaBox}

 \begin{FormulaBox}{Antiderivative of power other than $-1$}
  For $n \neq -1$,
  \begin{equation*}
   \IndefiniteIntegral{ x^n }{x}
   =
   \frac{1}{n+1} x^{n+1} + C
  \end{equation*}
 \end{FormulaBox}

 \begin{FormulaBox}{Antiderivative of power $-1$}
  \begin{equation*}
   \begin{split}
     \IndefiniteIntegral{ x^{-1} }{x}
     &=
       \LN{\Abs{x}} + C
     \\
     \IndefiniteIntegral{ \frac{1}{x} }{x}
     &=
       \LN{\Abs{x}} + C
   \end{split}
  \end{equation*}
 \end{FormulaBox}

 \begin{FormulaBox}{Antiderivative of exponential}
  \begin{equation*}
   \int \me^{A\,x} \,\dif{x}
   = \frac{1}{A} \me^{A\,x} + C
  \end{equation*}
 \end{FormulaBox}

\end{multicols}

%%% Local Variables:
%%% mode: latex
%%% TeX-master: "Business-calculus-workbook"
%%% End:
